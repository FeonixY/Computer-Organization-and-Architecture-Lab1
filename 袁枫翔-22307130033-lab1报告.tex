\documentclass[UTF8]{article}
\usepackage{blindtext}
\usepackage[T1]{fontenc}
\usepackage[utf8]{inputenc}
\usepackage{multirow}
\usepackage{booktabs}
\usepackage{amssymb}
\usepackage{ctex}
\usepackage{tikz}
\usepackage{listings}
\usepackage[fleqn]{amsmath}
\usepackage{xcolor}
\usepackage{color}
\usepackage{xcolor}
\usepackage{graphicx}
\usepackage{epstopdf}
\definecolor{dkgreen}{rgb}{0,0.6,0}
\definecolor{gray}{rgb}{0.5,0.5,0.5}
\definecolor{mauve}{rgb}{0.58,0,0.82}
\lstset{frame=tb,
	language=C, % 使用的语言
	aboveskip=3mm,
	belowskip=3mm,
	showstringspaces=false, % 仅在字符串中允许空格
	backgroundcolor=\color{white},   % 选择代码背景,必须加上\ usepackage {color}或\ usepackage {xcolor}
	columns=flexible,
	basicstyle = \ttfamily\small,
	numbers=none, % 给代码添加行号,可取值none, left, right.
	numberstyle=\small \color{gray},  % 行号的字号和颜色
	keywordstyle=\color{blue},
	commentstyle=\color{dkgreen}, % 设置注释格式
	stringstyle=\color{mauve},
	breaklines=true,   % 设置自动断行.
	breakatwhitespace=true, % 设置是否当且仅当在空白处自动中断.
	escapeinside=``, %逃逸字符(1左面的键),用于显示中文
	frame=single, %设置边框格式
	extendedchars=false, %解决代码跨页时,章节标题,页眉等汉字不显示的问题
	xleftmargin=0em,xrightmargin=1em, aboveskip=1em, %设置边距
	tabsize=4 % 将默认tab设置为4个空格
}
\begin{document}
	\begin{center}
	\textbf{{\huge Lab1 32-bit ALU设计}}
	\end{center}
	\section{实现方法}
	\begin{enumerate}
		\item [(1)] 概述
		\par 本项目使用了模块化的结构编写代码,分别编写了ALU,二进制转十进制,显示数字及符号等模块,最后在$top_module.sv$中对其进行整合。下面对每个模块进行讲解。
		\item [(2)] ALU
		\par 
		输入32位的一个数值和3位的一个操作数,输出32位的一个数值和4个标志码。根据不同操作数使用Verilog自带的逻辑运算,其中加减运算需要特殊设置标志码,其余运算都可用简单的运算判断标志码。
		\item [(2)] 显示模块
		\par 
		输入时钟信号,32位的三个数值,一个3位的操作数,输出AN和A2G信号。时钟取3位的信号,即数值在0~7之间循环。当时钟为0时显示第一个数的十位,时钟为1时显示第一个数的个位,时钟为2时显示运算符号,时钟为3时显示第二个数的十位,时钟为4时显示第二个数的个位,时钟为5时显示等于号,时钟为6时显示结果的十位,时钟为0时显示结果的个位。因为数码管运行很快,所以可以做到视觉上所有位同时亮起的效果。
		
		注:AbCdEFG分别代表与,或,加,与后数的反,或后数的反,减和SLT。
		\item [(3)] 顶层模块
		\par 
		将各个模块连接到一起,实现其功能,其中两个数值用高位28个0和4个波动开关表示。
	\end{enumerate}
	\section{实现体会}
	\par 在实现2进制转10进制时,我直接使用了verilog中的除法简易实现了功能,但查阅资料得知开发板并不自带除法器,verilog需要用现有元件自行拼凑出一个除法器,这无疑降低了运行速度,这一部分有待我通过后续学习进行改进。
\end{document}